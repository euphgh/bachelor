% !Mode:: "TeX:UTF-8"
\begin{acknowledgements}

大学本科四年一晃而过,回首走过的校园时光,心中倍感充实。
如今在我即将完成本科学业,踏入新的人生阶段之际,
我想向所有在我学术道路上给予支持和帮助的人们表达最衷心的感谢和敬意。

首先,我要衷心感谢我的校内毕业设计执导教师导师黄庆成,
感谢黄老师在我毕业设计期间的悉心指导和不懈支持。
作为我们大四时期专业课的教师,
我也从他的课程中学到了许多关于低功耗处理器的知识,并在毕业设计中得以实践和感悟。
同时,我也衷心感谢校外毕业设计指导教师张科、常轶松。
张老师和常老师的的专业知识、严谨治学的态度以及对学术研究的激情都对我产生了深远的影响。
在两位导师们的引领下,我不仅获得了处理器设计知识上的提升,更在芯片验证工具方面得到了进一步了解。
两位老师还为了解决我在设计中遇到的问题,帮忙向“香山”处理器设计团队的设计者们引荐我,
让我结识更多志同道合的学长,能够有机会向他们学习。
两位老师在我的论文研究中给予了宝贵的指导和建议,使我能够克服困难、突破瓶颈,并最终完成了这篇论文。
衷心感谢导师对我的悉心培养和教诲!

然后,我要衷心感谢本人进行毕业设计所在的课题组——先进计算机系统研究中心的工程组。
工程组拥有完备的FPGA、服务器等基础设计,这大大加快了我的开发效率。
工程组的学长学姐以及员工们,在我遇到困难时愿意倾听我的请求,不厌其烦的帮我答疑解惑。
甚至在下班期间,愿意带着我和他们一起行动:吃饭、打球、交流等,
在毕业设计的工作之余,能够放松身心。

其次,我要感谢本科期间的所有教师和学术导师,感谢他们的教诲和激励。
特别是舒燕君老师和刘国军老师,既是体系结构和操作系统专业课的任课教师,
也是参加“龙芯杯”——全国计算机系统能力大赛时的指导教师。
在参加竞赛期间为我提供了许多技术支持和物资保障,
没有你们的帮助,你们所传授的专科知识,我也无法完成本次毕业设计。
感谢你们的教导,使我能够在学术道路上不断成长。

衷心感谢我的家人,感谢他们对我学业的支持和理解。
他们是我坚强的后盾,在我遇到挫折和困难时给予我无尽的鼓励和支持。
他们的关爱和支持是我前进的动力,我会倍加珍惜。

最后,我还要感谢我的同学和朋友们,哈工大本部的龙同学,深圳校区的满同学、罗同学等。
十分感谢我们能够在哈尔滨工业大学开源技术协会中相遇。
正式因为遇到了你们,从你们身上学习学习了许多开源技术,我的工程能力才能长足进步,我才开始对开源社区报以浓厚兴趣。
不论是学习NixOS时的苦恼,亦或是合作写语言服务器时犯的错。
这些脚步一点一滴积累起来,现在的我才有勇气向“香山”、verilator项目提出Issue。
感谢你们与我一同走过本科生活的点点滴滴,我们互相鼓励、相互帮助,一起成长。
你们的友谊使我的大学生活更加丰富多彩,我会珍惜这段美好的回忆。

在本科生活即将结束之际,我要再次表达我的衷心感谢。
正是有了你们的支持和帮助,我才能顺利完成这篇毕业论文。
我将永远铭记你们的恩情,怀着感激之情,踏上新的征程。

谨向你们致以最诚挚的谢意!

衷心感谢!

\end{acknowledgements}
