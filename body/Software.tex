\chapter{系统软件下的初步评测}

\section{操作系统的适配与启动}

适配相关内容
\begin{itemize}
    \item FPGA平台:opensbi + Linux + 根文件系统
    \item 软件仿真平台:riscv-pk + Linux + 根文件系统
    \item kvmtool:启动虚拟机的用户态程序,如何获取,放入根文件系统
\end{itemize}

启动相关:描述取指卡住的问题,使用REMU重放波形得以解决

\section{操作系统下的性能评测}

新增一些用户态benchmark的测试实验,体现处理器原本的功能能正常工作

\chapter{虚拟机管理系统的调试}

\section{虚拟机启动的现象}
描述使用kvmtool启动虚拟机时的现象,以及使用REMU可以观察到的处理器内部情况

\section{现有调试手段的限制}
REMU无法直接观测处理器架构可见的状态,无法进行差分测试精确的定位出错现场

纯软件的模拟器+差分测试的方法速度太慢,对于深度运行的虚拟机是不可接受的

\section{潜在调试方案的探索}
软件仿真:使用模拟器快速运行,获取启动虚拟机时的检查点,让处理器硬件从该检查点运行。
硬件重放:1. Difftest硬件化;2. 把REMU检查点重放到verilator中进行差分测试
