\chapter{引言}
% 解释:RISC-V、开源处理器、高性能、虚拟化、虚拟化扩展
% 提及:云和数据中心领域是热的,在工业界有人不断尝试,是值得做的。
% 尽管该项目可能无法实现、但是需要去尝试、合并到工业界中。
% 以未来为切入点,画大饼。
% 历史片段

% 2018年11月8日,在网信办、中科院等多个国家部委支持和指导下,
% RISC-V中国联盟于浙江乌镇举行的第五届互联网大会上正式宣布成立。
% 从成立到现在的5年见,可以看见国内RISC-V相关产业在快速的发展。
% RISC-V是一个基于精简指令集原则的指令集架构,与x86、ARM架构是对应的关系。
% 然而与其他指令集最大的不同点在与开源性,即不需要收取高额的版权费,
% 另一方面,x86和ARM指令集知识产权的公司均为国外公司,不利于我国实现关键芯片的自主可控。
% 因此,开源意味着自由、安全及可控。
% 此外,RISC-V还具有简洁、模块化的特点,
% 意味着轻量化、低功耗、小体积,因此非常适用于移动设备,以及在多场景的灵活适应性。
% 正是具有以上特点,在计算机体系结构领域,RISC-V受到了学术界和工业界的广泛瞩目。
% 短短六年,中国目前有300家以上公司在关注RISC-V或以RISC-V指令集进行开发。
% 从嵌入式到AI服务器,各种类别的RISC-V处理器核百花齐放。
% 美国计算机学会电子数据图书馆(ACM digital library)中
% 随着RISC-V生态系统的不断完善和国内芯片厂商的不断努力,
% 我们有理由相信,未来RISC-V将在中国的芯片领域有更大的应用空间和市场份额。

% 中科院计算所研究员孙凝晖院士如此说道:“
% 开源模式不仅仅是一种商业模式,也是一种生态构建方法,是一种复杂系统开发方法。
% 更蕴含着一种精神。开源不仅仅公开源代码,更重要的是协作开发流程的建立与社区治理机制的建设。”

% 和中国知网上收录的关于RISC-V的文章在2018到2023年的平均增长比例达到了17.21\%和17.37\%

虚拟化技术,作为计算机科学领域的一项关键技术,为云数据中心提供的各类服务构建了核心的技术框架。
随着近年来云数据中心及云服务提供商的快速发展,从初期的虚拟机与容器技术,
到新兴的函数即服务(Function as a Service, FaaS)与无服务器计算(Serverless)模式,
均展现出了强烈的市场需求和广阔的发展潜力。
这一趋势不断促进虚拟化技术的进步与创新,尤其是在处理器虚拟化领域。

\begin{figure}[htbp]
\centering
\includegraphics[scale=0.5]{virtual.png}
\caption{云数据中心在不同指令集架构的发展}
\end{figure}

处理器虚拟化技术,作为实现硬件兼容性的重要手段之一,
使得虚拟机能够在与底层物理硬件架构完全不同的环境下运行不同操作系统,从而在云计算服务中扮演着至关重要的角色。
经过多年的深入研究,尤其是在需求驱动的影响下,
关于虚拟化技术及其评估方法的探讨,特别是在x86和ARM体系结构上,已经达到了高度成熟的阶段。
然而,当转向RISC-V,一个新兴的开源精简指令集,发现其处理器虚拟化技术的研究仍处于起步阶段,研究潜力巨大。

RISC-V凭借其简洁、高效以及模块化的特性,在计算机体系结构领域受到了学术界和工业界的广泛瞩目。
为了实现处理器虚拟化,RISC-V推出了一种新的指令集扩展——虚拟化扩展(Hypervisor Extension)。
该扩展明确规定了实现处理器虚拟化所必需的硬件功能,其中包括特权指令和页表翻译等关键技术。

关于RISC-V虚拟化扩展的研究,加州大学的伯克利分校迈出了第一步。
他们研发了Rocket Chip,一个开源的RISC-V顺序流水线处理器,
并在其上实现了完整的虚拟化扩展,进行了一系列虚拟化相关的评测。
但是在真实应用场景下,处理器的情况会更为复杂,多核乱序的处理器才是虚拟化服务器的主流。
关于这方面的研究尚未完全展开,亟待进行。

因此,本项目尝试在“香山”处理器中,实现虚拟化扩展。
香山是由中国科学院计算技术研究所牵头发起高性能开源RISC-V处理器项目,
是目前国际上性能最高的开源超标量乱序RISC-V处理器核。
具体而言,本项目主要的研究工作如下:

\begin{itemize}
\item 在“香山”处理器硬件设计中实现虚拟化扩展
\item 为“香山”的虚拟化扩展版本适配虚拟化系统软件
\item 在FPGA中对“香山”的虚拟化版本的进行性能评测
\end{itemize}

关于虚拟化系统软件的适配,本项目采用的是Linux KVM,一个基于Linux的Type2虚拟机管理系统。
在KVM启动虚拟机的过程中,由于软件和硬件代码的复杂性,在软硬件联合调试时会出现一些难以解决的错误。
此时离处理器开始运行已经经过了很长的时钟周期,使用常规的调试手段难以追踪错误的来源。
逻辑软件仿真速度的限制、FPGA平台无法定位错误现场的限制等,逐渐成为调试处理器运行大型系统程序的最大困难。
因此,本文也对“香山”处理器的虚拟化扩展调试手段进行了探索,
分别在软件逻辑仿真和FPGA硬件加速仿真的两个方向上进行了一些尝试。