% !TEX root = ../main.tex
\chapter{绪论}[Introduction]
虚拟化技术,作为计算机科学领域的一项关键技术,为云数据中心提供的各类服务构建了核心的技术框架。
随着近年来云数据中心及云服务提供商的快速发展,从初期的虚拟机与容器技术,
到新兴的函数即服务(Function as a Service, FaaS)与无服务器计算(serverless computing)模式,
均展现出了强烈的市场需求和广阔的发展潜力。
这一趋势不断促进虚拟化技术的进步与创新,尤其是在处理器虚拟化领域。
处理器虚拟化技术,作为实现硬件兼容性的重要手段之一,使得虚拟机能够在与底层物理硬件架构完全不同的环境下运行不同操作系统,从而在云计算服务中扮演着至关重要的角色。

经过多年的深入研究,尤其是在需求驱动的影响下,
关于虚拟化技术及其评估方法的探讨,特别是在x86和ARM体系架构上,已经达到了高度成熟的阶段。
然而,当我们转向RISC-V,这个新兴的开源精简指令集,我们发现其处理器虚拟化技术的研究仍处于起步阶段,研究潜力巨大。RISC-V凭借其简洁、高效以及模块化的特性,在计算机体系架构领域受到了学术界和工业界的广泛瞩目。
为了实现处理器的虚拟化,RISC-V推出了一种新的指令集扩展——虚拟化扩展(Hypervisor Extension)。
该扩展明确规定了实现处理器虚拟化所必需的硬件功能,其中包括特权指令和页表翻译等关键技术。

关于RISC-V虚拟化扩展的研究,加州大学的伯克利分校迈出了第一步。
他们研发了Rocket Chip,一个开源的RISC-V顺序流水线处理器,
并在其上实现了完整的虚拟化扩展,进行了一系列虚拟化相关的评测。
但是在真实应用场景下,硬件情况会更为复杂,
多核乱序的处理器核心才是虚拟化服务器的主流。
关于这方面的研究尚未完全展开,亟待进行。