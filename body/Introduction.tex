% !TEX root = ../main.tex
\chapter{简介}[Introduction]
虚拟化技术,作为计算机科学领域的一项关键技术,为云数据中心提供的各类服务构建了核心的技术框架。
随着近年来云数据中心及云服务提供商的快速发展,从初期的虚拟机与容器技术,
到新兴的函数即服务(Function as a Service, FaaS)与无服务器计算(serverless computing)模式,
均展现出了强烈的市场需求和广阔的发展潜力。
这一趋势不断促进虚拟化技术的进步与创新,尤其是在处理器虚拟化领域。
处理器虚拟化技术,作为实现硬件兼容性的重要手段之一,
使得虚拟机能够在与底层物理硬件架构完全不同的环境下运行不同操作系统,从而在云计算服务中扮演着至关重要的角色。

经过多年的深入研究,尤其是在需求驱动的影响下,
关于虚拟化技术及其评估方法的探讨,特别是在x86和ARM体系架构上,已经达到了高度成熟的阶段。
然而,当转向RISC-V,一个新兴的开源精简指令集,发现其处理器虚拟化技术的研究仍处于起步阶段,研究潜力巨大。
RISC-V凭借其简洁、高效以及模块化的特性,在计算机体系架构领域受到了学术界和工业界的广泛瞩目。
为了实现处理器虚拟化,RISC-V推出了一种新的指令集扩展——虚拟化扩展(Hypervisor Extension)。
该扩展明确规定了实现处理器虚拟化所必需的硬件功能,其中包括特权指令和页表翻译等关键技术。

关于RISC-V虚拟化扩展的研究,加州大学的伯克利分校迈出了第一步。
他们研发了Rocket Chip,一个开源的RISC-V顺序流水线处理器,
并在其上实现了完整的虚拟化扩展,进行了一系列虚拟化相关的评测。
但是在真实应用场景下,处理器的情况会更为复杂,多核乱序的处理器才是虚拟化服务器的主流。
关于这方面的研究尚未完全展开,亟待进行。

因此,本项目尝试在“香山”处理器中,实现虚拟化扩展。
香山是由中国科学院计算技术研究所牵头发起高性能开源RISC-V处理器项目,
是目前国际上性能最高的开源超标量乱序RISC-V处理器核。
具体而言,本项目主要的研究工作如下:

\begin{itemize}
\item 在“香山”处理器硬件设计中实现虚拟化扩展
\item 为“香山”的虚拟化扩展版本适配虚拟化系统软件
\item 在FPGA中对“香山”的虚拟化版本的进行性能评测
\end{itemize}

关于虚拟化系统软件的适配,本项目采用的是Linux KVM,一个基于Linux的Type2虚拟机管理系统。
在KVM启动虚拟机的过程中,由于软件和硬件代码的复杂性,在软硬件联合调试时会出现一些难以解决的错误。
此时离处理器开始运行已经经过了很长的时钟周期,使用常规的调试手段难以追踪错误的来源。
逻辑软件仿真速度的限制、FPGA平台无法定位错误现场的限制等,逐渐成为调试处理器运行大型系统程序的最大困难。
因此,本文也对“香山”处理器的虚拟化扩展调试手段进行了探索,
分别在软件逻辑仿真和FPGA硬件加速仿真的两个方向上进行了一些尝试。