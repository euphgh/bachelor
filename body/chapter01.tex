% !TEX root = ../main.tex

% 中英标题:\chapter{中文标题}[英文标题]
\chapter{绪论}[Introduction]

\section{课题背景及研究的目的和意义}[Background, objective and significance of the subject]

虚拟化是一项重要的计算机技术,为云数据中心各类服务提供关键技术支撑。
近年来,云数据中心和云服务提供商发展迅速。
从传统的虚拟机和容器,到新兴的函数服务(Function as a service)和无服务计算(serveless),
都有着巨大的需求和广阔的发展前景。
这不断推动着虚拟化技术的发展,特别是处理器虚拟化技术。
处理器虚拟化技术作为虚拟化中硬件兼容性最完整的一项技术,
能够使虚拟机可以运行与底层硬件完全不同的操作系统,在云服务中发挥着重要作用。

多年来,在需求的刺激下,关于虚拟化技术和评测手段的研究,特别是x86和ARM架构,已经相当成熟。
但是RISC-V,作为一个新兴的、开源的精简指令集,其处理器虚拟化技术的研究尚未完全展开,还有巨大的研究空间。
由于具有简洁、高效、模块化的特点,RISC-V在计算机体系结构领域中受到了学术界和工业界广泛关注。
为了实现处理器虚拟化,RISC-V基金会于2021.12月提出一个指令集扩展:虚拟化扩展(Hypervisor Extension)。
虚拟化扩展作为指令集架构的子集,规定了实现处理器虚拟化必须添加的硬件功能,包括特权指令和页表翻译等。

关于RISC-V虚拟化扩展的研究,加州大学的伯克利分校迈出了第一步。
他们研发了Rocket Chip,一个开源的RISC-V顺序流水线处理器,
并在其上实现了完整的虚拟化扩展,进行了一系列虚拟化相关的评测。
但是在真实应用场景下,硬件情况会更为复杂,
多核乱序的处理器核心才是虚拟化服务器的主流。
关于这方面的研究尚未完全展开,亟待进行。



\section{本文的主要研究内容}
该项目主要研究对象是RISC-V开源处理器,研究内容是硬件虚拟化实现和评测。
本课题瞄准XiangShan项目的“南湖”架构处理器。
XiangShan是由中国科学院计算技术研究所牵头发起高性能开源RISC-V处理器项目,
是目前国际上性能最高的开源高性能 RISC-V 处理器核。
而“南湖”则是一个较为稳定的用于学术研究的版本。