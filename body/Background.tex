% !TEX root = ../main.tex
\chapter{技术背景及研究动机}[Background]

\section{RISC-V虚拟化扩展}
RISC-V的虚拟化扩展需要添加的硬件功能大致可分为特权控制和虚拟内存两大部分。
在特权控制方面,主要新增了3种特权模式、12条特权指令、23个控制状态寄存器以及虚拟化相关的4种中断和6种异常。
在虚拟内存方面,首先定义了第二阶段地址翻译的概念,
同时添加了上述部分的控制状态寄存器、虚拟内存管理指令和异常,用于辅助程序员管理第二阶段地址翻译。

\paragraph{特权控制}
在未支持虚拟化扩展时,特权等级只有机器模式(M-mode)、监管模式(S-mode)和用户模式(U-mode)。
分别对应标准体系结构中运行bootloader,操作系统和用户程序的机器状态。
然而,在虚拟化扩展下,原本的监管模式被改成虚拟机管理模式(HS-mode),对应于虚拟机管理程序的层级。
还新增了虚拟监管模式(VS-mode)和虚拟用户模式(VU-mode),对应于虚拟机和虚拟机下的用户态程序。
此外还保留了原始的用户模式,对应于Type2虚拟机监管系统下的宿主操作系统的用户程序。
例如在Linux KVM中,需要先启动宿主Linux操作系统,处理器运行在虚拟机管理模式下。
在该宿主操作系统在不启动KVM时,和一般的
通过在硬件中添加Virtual位到原始的特权级别中,代表是否是虚拟机模式还是虚拟机管理模式,用于区分HS和VS,U和VU。
除了特权级,还需要添加相关的RISC-V控制状态寄存器(Control Status Register)是虚拟机管理模式(HS-mode)下的hstatus、hdeleg、hvip等,他们用于处理虚拟机管理程序的中断委托、使能。
处理器捕获相关敏感指令的能力。

第二,虚拟机级(VS-mode)的vsstatus、vstvec、vsatp等寄存器,用于处理虚拟机中的异常自陷的流程,比如配置自陷地址、保存异常指令的PC等。他们作为原始监管模式(S-mode)的副本,需要重新创建。第三、添加虚拟机管理级(HS-mode)相关的特权指令,包括页表数据结构同步指令HFENCE.VVMA/GVMA,这部分和内存虚拟化相关,会在下一部分详细说明。还包括虚拟机管理级访存指令,在虚拟机管理模式下(HS-mode)的访存只会进行第二阶段的地址翻译,只有在虚拟机模式下(VS-mode)才会使用两级页表转换。而特权指令HLV.width, HLVX.HU/WU, HSV.width能够在虚拟机管理模式下使用两级页表翻译访问使用特定虚拟机的虚地址进行内存访问,用于在虚拟机自陷到虚拟机管理程序时方便处理异常。最后,需要添加虚拟机管理级和虚拟机级的异常和中断,包括VS-mode的外部中断、时钟中断、软件中断等,HS-mode下的所有中断继承原始的S-mode下所有的中断的前提下,还需要添加一个虚拟机外部中断,用于实现虚拟机直通中断。异常部分需要添加从虚拟机下(VS-mode)自陷(Ecall)到虚拟机管理系统(HS-mode)。还有相关的第二阶段地址翻译页错误(guest page fault)。

\paragraph{虚拟内存}

\section{虚拟化扩展实现案例}
虚拟化扩展的相关工作介绍

\section{“香山”处理器的内存管理单元}
介绍未实现虚拟化扩展的虚拟内存单元,方便后续对比

\section{处理器设计的调试及评测平台}
介绍本文中出现的调试和评测平台

调试:逻辑分析仪、仿真工具(verilator、iverilog),辅之以模拟器(QEMU、Spike)和差分测试方法,更精确定位错误现场。

评测:硬件部署在FPGA中进行原型验证

也有通过FPGA加速仿真的方式(REMU),能够更高效的得到波形进行调试