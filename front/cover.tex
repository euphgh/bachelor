% !Mode:: "TeX:UTF-8"

\hitsetup{
  %******************************
  % 注意:
  %   1. 配置里面不要出现空行
  %   2. 不需要的配置信息可以删除
  %******************************
  %
  %=====
  % 秘级
  %=====
  statesecrets={公开},
  natclassifiedindex={TM301.2},
  intclassifiedindex={62-5},
  %
  %=========
  % 中文信息
  %=========
% ctitleone={高性能RISC-V开源处理},%本科生封面使用
% ctitletwo={器虚拟化扩展与系统评测},%科生封面使用
  ctitlecover={高性能RISC-V开源处理器\\虚拟化扩展与系统评测},%放在封面中使用,自由断行
  ctitle={高性能RISC-V开源处理器虚拟化扩展与系统评测},%放在原创性声明中使用
  % csubtitle={一条副标题}, %一般情况没有,可以注释掉
  cxueke={工学},
  csubject={计算机科学与技术},
  caffil={计算学部},
  cauthor={胡光辉},
  csupervisor={黄庆成副教授},
  cassosupervisor={某某某教授}, % 副指导老师
  ccosupervisor={某某某教授}, % 联合指导老师
  % 如果是深圳本科毕业论文,需要取消注释下一行,并将内容改为“规范”中要求的封面第一页最下方的日期
  szshortcdate={2024年6月},
  % 日期自动使用当前时间,若需指定按如下方式修改:
  cdate={2024年5月},
  cstudentid={120L052208},
  % cstudenttype={学术学位论文}, %非全日制教育申请学位者
  % cnumber={no9527}, %编号
  % cpositionname={哈铁西站}, %博士后站名称
  % cfinishdate={20XX年X月---20XX年X月}, %到站日期
  % csubmitdate={20XX年X月}, %出站日期
  % cstartdate={3050年9月10日}, %到站日期
  % cenddate={3090年10月10日}, %出站日期
  %(同等学力人员)、(工程硕士)、(工商管理硕士)、
  %(高级管理人员工商管理硕士)、(公共管理硕士)、(中职教师)、(高校教师)等
  %
  %
  %=========
  % 英文信息
  %=========
etitle={High performance RISC-V open source processor hypervisor extension and systm evaluation},
  esubtitle={This is the sub title},
  exueke={Engineering},
  esubject={Computer Science and Technology},
  eaffil={\emultiline[t]{School of Mechatronics Engineering \\ Mechatronics Engineering}},
  eauthor={Yu Dongmei},
  esupervisor={Professor XXX},
  eassosupervisor={XXX},
  % 日期自动生成,若需指定按如下方式修改:
  edate={December, 2017},
  estudenttype={Master of Art},
  %
  % 关键词用“英文逗号”分割
ckeywords={RISC-V, 高性能处理器, 虚拟化, 系统评测},
ekeywords={RISC-V, high-performance processor, virtualization, system evaluation},
}

\begin{cabstract}
本文旨在研究RISC-V指令集的虚拟化扩展,以高性能开源处理器“香山”为基础,构建了一个支持虚拟化的软硬件系统。
虚拟化是一项在云服务和数据中心中广泛使用的技术。
在需求的推动下,主流计算机体系结构已纷纷实现了虚拟化的硬件扩展支持。
RISC-V作为一种新兴的、在物联网终端设备取得一定成果的开源指令集,也开始关注云数据中心的需求,
于是提出了处理器的虚拟化扩展规范,为研究提供了巨大的空间。
关于本文的研究方法,先在硬件级别对处理器进行虚拟化扩展的微架构设计,
然后基于硬件平台对虚拟机管理软件进行适配,尝试在处理器中启动管理软件和虚拟机。
最后对软硬件整体进行系统级评测。基于以上的研究方法得到如下结果:

在处理器微架构方面,
虚拟化扩展的实现主要需要扩展了“香山”的特权管理单元和虚拟内存单元。
在虚拟内存单元方面,本文采用了分布式的一级页表缓存和集中式的地址翻译引擎,以平衡翻译速度和时序压力。
通过添加第二阶段地址翻译单元、实现了第二阶段页表翻译。
还通过复用二级页表缓存、猝发传输获取页表、压缩一级页表缓存表项等的微架构加快了翻译速度。

在虚拟机管理软件方面,
本文在添加虚拟化扩展后的“香山”上启动了Linux,并对主机性能进行系统级评测。
同时,对调试工具进行了探索,包括基于纯软件仿真的差分测试框架和基于FPGA的仿真加速平台。
这是因为在Linux中使用KVM启动虚拟机时,由于处理器潜在的错误而启动失败。
软硬件联合调试的复杂性以及现有调试工具的能力限制使错误难以解决。
为调试错误,本文尝试了在差分测试系统中使用体系结构检查点,帮助跳过仿真操作系统启动的冗长时间。
还尝试扩展FPGA加速仿真平台获取体系结构信息的能力,力求在高效仿真的基础上实现精确定位出错现场。

根据以上结果可得出结论:
本文实现了一个RISC-V指令集的、支持处理器虚拟化扩展的软硬件系统,
该系统以高性能乱序处理器“香山”为底层硬件是,以Linux-KVM为系统软件。
虚拟化扩展对的处理器在操作系统中的主机性能影响微乎其微。
处理器硬件设计的现有调试工具在虚拟化场景下的调试能力有限,
亟需一种既能获取体系结构信息又能快速仿真调试平台。

\end{cabstract}

\begin{eabstract}
The purpose of this paper is to
investigate the virtualization extension of the RISC-V instruction set
and build a software-hardware system that supports virtualization
based on the high-performance open-source processor "Xiangshan".
Virtualization is a widely used technology in cloud services and data centers.
Driven by demand, mainstream computer architectures have implemented hardware extension support for virtualization.
RISC-V, as an emerging open-source instruction set that has achieved certain results in IoT terminal devices,
has also started to pay attention to the needs of cloud data centers.
Therefore, it proposed the virtualization extension specification for processors,
providing significant research opportunities.
Regarding the research methodology of this paper,
the microarchitecture design of processor virtualization extension is first carried out at the hardware level.
Then, the virtual machine management software is adapted based on the hardware platform,
and attempts are made to launch the management software and virtual machines on the processor.
Finally, a system-level evaluation of the overall software and hardware is conducted.
Based on the above research methodology, the following results are obtained:

In terms of processor microarchitecture,
the implementation of virtualization extension mainly requires the extension of Xiangshan's privilege management unit and virtual memory unit.
For the virtual memory unit, distributed first-level page table cache and a centralized address translation engine
are applied to balance translation speed and timing pressure.
By adding a second-stage address translation unit, the second-stage page table translation is realized.
The translation speed is accelerated through microarchitectural techniques such as reusing the second-level page table cache,
burst transfers for obtaining page tables,
and compressing first-level page table cache entries.

In terms of virtual machine management software,
Linux on the Xiangshan processor with virtualization extension
and conducts a system-level performance evaluation of the host.
Additionally, exploration is carried out for debugging tools,
including a differential testing framework based on pure software simulation and a simulation acceleration platform based on FPGA.
This is because, when launching virtual machines using KVM in Linux, startup failures may occur due to potential processor errors.
The complexity of software-hardware joint debugging
and the limitations of existing debugging tools' capabilities
make error resolution difficult.
To debug errors, architectural checkpoints in the differential testing system is attempted to skip the lengthy process of simulating operating system boot.
It also attempts to extend the capabilities of the FPGA acceleration simulation platform to obtain architectural information,
aiming to achieve precise localization of error site when keeping fast simulation speed.
  
Based on the above results,
the conclusion can be drawn that a software-hardware system that supports the virtualization extension of the RISC-V instruction set was implemented.
The system utilizes the high-performance out-of-order processor Xiangshan as the underlying hardware and Linux-KVM as the system software.
The virtualization extension has negligible impact on the host performance of the processor in the operating system.
The existing debugging tools for processor hardware design have limited debugging capabilities in virtualization scenarios,
highlighting the urgent need for a platform that can obtain architectural information and enable fast simulation debugging.
\end{eabstract}
